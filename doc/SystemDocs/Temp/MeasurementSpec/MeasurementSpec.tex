\documentclass[english]{article}
\usepackage[T1]{fontenc}
\usepackage[latin1]{inputenc}
\usepackage{geometry}
\usepackage{longtable}
\usepackage{paralist}


\geometry{verbose,letterpaper,tmargin=1in,bmargin=1in,lmargin=1in,rmargin=1in}
\usepackage{graphicx}

\oddsidemargin  0.0in \evensidemargin 0.0in \textwidth      6.5in
\headheight     0.25in \topmargin      0.0in \textheight=8.5in
\makeatletter
\usepackage{babel}
\makeatother
\begin{document}




\section{Overview}

GMAT models measurements as either a primitive measurement type, a
compound measurement type (sequence of primitives), or a custom
measurement type that is specific to an individual system or is
proprietary. Primitive measurements include range, range rate,
Doppler and angles measurements.  For each primitive type, models
are provided for different sensor combinations.  For example, an
angle measurement between a telescope and a target is modelled
differently than an angles measurement between a Radar and a target.

For compound and custom measurements, solving for the measurement
value and its partial derivatives involves locating discrete events
involved in the measurement process.  This in turn requires
knowledge of the states and STM over portions of the event duration.
Hence, in general, real-world measurements are complex and involve
the interaction of several system components to evaluate.  In the
following pages, we present a detailed example of a how GMAT
evaluates a real world measurement.

Assume the measurement we wish to evaluate is the average range
rate.

\section{Mathematical Model}

For compound and custom measurements, solving for the measurement
value and its partial derivatives involves locating discrete events
involved in the measurement process.  This in turn requires
knowledge of the states and STM over portions of the event duration.
Hence, in general, real-world measurements are complex and involve
the interaction of several system components to evaluate. In the
following pages, we present a detailed example of a how GMAT
evaluates a real world measurement.
%
\begin{figure}[h!]
\centerline{
\begin{picture}(100,475)
\special{psfile= RangeRateFig.eps hoffset= -135 voffset= 30
hscale=65 vscale=65}
        \makebox(300,590){$\mathbf{r}_{g}(t)$}
        \makebox(-430,580){$\mathbf{r}_{s}(t - \tau_1)$}
        \makebox(-590,580){$\mathbf{r}_{s}(t - \tau_2)$}
        \makebox(-730,590){$\mathbf{r}_{g}(t - \tau_3)$}
        \makebox(-680,850){$\rho_u$}
        \makebox(-380,850){$\rho_d$}
\end{picture}}\vskip -4.0 in
    \caption{ Illustration of Average Range Rate Measurement}
    \label{Fig:RangeRate}
\end{figure}

\begin{itemize}
   \item $\rho_d$ is the downlink range
   \item $\rho_u$ is the uplink range
   \item $t$ is the measurement anchor time
   \item $\Delta t_j$ is the time offset, from $t$ for the
   $j^{th}$ subevent in the measurement process.
   \item $\mathbf{r}_s$ is the spacecraft position vector
   \item $\mathbf{r}_g$ is the ground station position vector
   \item $\Delta t_a$ is the known averaging interval
\end{itemize}

\begin{Large}Suggested steps in measurement process:\end{Large}
%
\begin{enumerate}
    \item Prop all participants to measurement anchor time $t$
    \item Save state and STM
    \item Locate epoch of event at $\tau_1$
    \item Save state and STM
    \item Locate epoch of event at $\tau_2$
    \item Save state and STM
    \item Locate epoch of event at $\tau_3$
    \item Save state and STM
    \item Calculate measurements and partials using the following
    equations.
\end{enumerate}

\begin{equation}
    \bar{\dot{\rho}} = \frac{\rho_d - \rho_u}{\tau_2 - \tau_1}
\end{equation}
%
where
%
\begin{equation}
    \rho_d = \|\mathbf{r}_g(t)  - \mathbf{r}_s(t - \tau_1)\|
\end{equation}
%
\begin{equation}
    \rho_u = \|\mathbf{r}_s(t - \tau_2)  - \mathbf{r}_g(t - \tau_3)\|
\end{equation}
%
The partial derivatives of the measurement w/r/t $\mathbf{r}_s(t)$,
$\mathbf{v}_s(t)$ and $\mathbf{r}_g(t)$ are
%
\begin{eqnarray}
    \frac{\partial \bar{\dot{\rho}} }{\partial\mathbf{r}_s(t) } &=& \frac{1}{\tau_2 - \tau_1}\left( \hat{\rho}_u^T\mathbf{A}_s(t-\tau_2,t)
    +  \hat{\rho}_d^T\mathbf{A}_s(t-\tau_1,t) \right)\\
    %
    \frac{\partial \bar{\dot{\rho}} }{\partial\mathbf{v}_s(t) } &=& \frac{1}{\tau_2 - \tau_1}\left( \hat{\rho}_u^T\mathbf{B}_s(t-\tau_2,t)
    +  \hat{\rho}_d^T\mathbf{B}_s(t-\tau_1,t) \right)\\
    %
    \frac{\partial \bar{\dot{\rho}} }{\partial\mathbf{r}_g(t) } &=& \frac{1}{\tau_2 - \tau_1}\left( \hat{\rho}_u^T\mathbf{B}_g(t-\tau_3,t)
    +  \hat{\rho}_d^T \right)
\end{eqnarray}

The event function for the downlink and uplink legs are
respectively:
%
\begin{equation}
   \mathcal{E}_d = \| \mathbf{r}_g(t) - \mathbf{r}_s(t - \tau_1)\| - c \tau_1 = 0 \hspace{.25 in} \mbox{(Solve for $\tau_1$)}
\end{equation}
%
\begin{equation}
   \mathcal{E}_u = \| \mathbf{r}_g(t - \tau_3) - \mathbf{r}_s(t - \tau_2)\| - c (\tau_3 - \tau_2) = 0 \hspace{.25 in} \mbox{(Solve for $\tau_3$)}
\end{equation}
%
where we know that $\tau_2 = \tau_1 - \Delta t_a$.

\end{document}

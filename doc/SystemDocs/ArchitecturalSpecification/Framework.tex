\chapter{GMAT System Framework}

\chapauthor{Darrel J. Conway}{Thinking Systems, Inc.}

The GMAT system consists of a high level framework, the GMAT Application, that manages system level
messages processed by GMAT.  This framework contains a single instance of the core GMAT executive,
the Moderator, which manages the functionality of the system.  The Moderator interacts with five
high level elements, shown in Figure 1, that function together to run the system.

The Interpreter subsystem consists of two separate components.  The interpreter contains the current
mission script, used to generate the mission event sequence, and the interface to the GMAT user
interface.  This latter interface takes the defined user actions and passes these actions to the
appropriate elements in the system -- for instance, when a user presses a ``Run Mission'' button on
the GMAT GUI, the command is passed to the user action interpreter, which then configures the
objects needed to run the current script and then starts the execution of the script.  (Okay, that
sentence assumes a lot that I haven't talked about yet...)

The Environment subsystem contains configuration information for the system data files, external
programs (e.g. MATLAB), and a number of utility subsystems (Okay, I needed someplace to put these --
is this the best place?) used to perform common tasks.  It acts as the repository for all of the
information needed for GMAT to talk to other elements running on a user's workstation, along with
the central location for information about the data files used by the system.

GMAT contains numerous classes that are used to perform spacecraft modeling.  These classes are all
managed by a set of components that construct instances of the classes needed by a script; these
components are shown on Figure 1 as a set of object factories defined for the system.  This
infrastructure provides the flexibility needed by the system to give users the ability to add custom
components to the system, and will be described in more detail later in this document.  (This piece
is pretty core to the design I'm thinking about right now, so it needs to be examined closely to be
sure we get what we want in GMAT.  Of course, that means it's also the hardest part to explain --
especially when I try to muddle through the way the system puts it together with the script
interpreter and the configurations!)

The GMAT Moderator includes a container used to manage lists of configured components used by the
system to perform mission analysis.  GMAT maintains these lists as the core object structures
manipulated by the system to perform mission analysis.  The Moderator has the following core lists
used in a mission timeline:

\begin{enumerate}
\item Solar System Configurations: A list of the celestial objects (star(s), planets, moons,
asteroids and comets that represent the playing field for mission analysis scenarios
\item Propagation Configurations: A list of configured propagation elements used to evolve the
modeled elements during analysis
\item Asset Configurations: A container for spacecraft, formations1 of spacecraft, and ground assets
\item Force Model Configurations: Collections of forces used to model perturbations acting on the
assets
\item Script Configurations: Either complete timelines or ``subscripts'', consisting of a sequence
of actions taken by the system to model all or a piece of an orbit problem
\item Mathematical Configurations: Elements used to perform custom calculations and for
communication with external programs like MATLAB
\end{enumerate}

GMAT runs are performed in the GMAT Sandbox.  This portion of the system is the container for the
components of a run, linked together to perform the sequence of events in the mission timeline. When
a user tells GMAT to run the script, the system moderator uses the script interpreter to interpret
the contents of a script, and to place the corresponding script elements into the sandbox for use
during the run.  The Moderator links each element placed into the sandbox to its neighboring
elements.  Once the full script has been translated into the components in the sandbox, the
Moderator starts the run by calling the Run method on the sandbox.

The following sections describe each of these components more completely.  These descriptions are
followed by several sample system configurations.  The last section of this document provides
details of the classes used in this design.




\chapter{Preface}

Welcome to the programming side of the General Mission Analysis Tool, GMAT!  This document
describes the design of the GMAT system, starting from an overview of the requirements for the
system and the architecture built to meet these requirements, and proceeding through descriptions
of the design for the components that fit into this architecture.

The purpose of this document is to give the reader an understanding of the design goals and
implementation of GMAT.  It is written to prepare you to work with GMAT at a source code level.  In
this document we present the rationale for GMAT's architectural design based on the requirements for
the system, and then construct the architecture based on that rationale.

The architectural framework is presented taking a top-down approach.  First we define a way to think
about GMAT's structure in terms of high level functionality, grouped into logical packages.  Then we
examine key elements of these packages, and explain how they interact to complete a few typical
tasks.  With a few exceptions, we do not document the details of the classes and objects in the
system.  That task is left to the GMAT API, generated using the Doxygen\cite{doxygen} open source
tool.

\section*{Intended Audience}

This document is written primarily for people interested in working with GMAT's source code, either
to enhance the system by adding new components, to debug existing features in a way consistent
with GMAT's design, to gain insight into pieces of the system that they may want to use elsewhere,
or to learn how GMAT was assembled to help design a similar system.  The primary audience for this
document is the software development community -- programmers, system analysts, and software
architects.

Analysts that are interested in understanding how GMAT performs its tasks can gain an understanding
of the system by reading the first few chapters of this document.  If some element of GMAT is not
behaving the way you expect, you might also want to look up the description of that object in later
chapters.  In addition, many of the details about how calculations are performed in GMAT are
contained in the Mathematical Specifications\cite{mathSpec}.  If you are more interested in
understanding how to use GMAT as an analyst, you might want to read the User's Guide\cite{userGuide}
rather than this document.

\section*{Assumed Background}

The GMAT design was developed using object-oriented technologies.  The design is presented using
Unified Modeling Language (UML) diagrams, using a style similar to that presented in \textbf{UML
Distilled}\cite{fowler}.  You can find a description of the use of UML diagrams as used in this
document in Appendix~\ref{chapter:UMLDiagrams}.  While you don't need to be an expert in either of
these fields to understand the content presented here, you will benefit from some preliminary
reading at an introductory level.

The GMAT design leverages several industry standard design patterns.  The patterns used are
summarized in Appendix~\ref{chapter:Patterns}.  If you are unfamiliar with the design pattern
literature, you'd benefit from reading -- or at least skimming -- some of the standard texts (see,
for example, \textbf{Design Patterns}\cite{GoF}).

GMAT is written in C++.  On the rare occasions that actual code is presented in this document, that
code is in C++.  As you go deeper into the GMAT's design, the underlying coding language will
become more important.  Therefore, if you plan to work with the GMAT source code, you'll need to
have an understanding of the C++ programming language.

In addition, the standard GMAT GUI is written using the wxWidgets\cite{wxWidgets} GUI toolkit.  If
you plan to work with GMAT's GUI code, you'll want to bo some preliminary exploration of wxWidgets.
 A good place to start is the wxWidgets book\cite{wxBook}, which, while slightly out of date at
this writing, does present a rather complete description of wxWidgets.

\section*{Useful Preliminaries}

This document describes the GMAT source code -- sometimes at a very high level, but also at times at
a rather low level of detail.  You'll benefit from having a copy of the current source available
for viewing at times when the descriptions found here are not as clear as you'd like.  You can
retrieve the source code either at the GMAT website
(http://gmat.gsfc.nasa.gov/downloads/source.html) or from the download pages or the code repository
at SourceForge (http://sourceforge.net/projects/gmat).

This document does not describe the detailed design of every class in GMAT, in part because the
resulting document would be extremely large, but also because GMAT is an evolving system.  New
features are being added to the program as the system grows, so the best view of the details of GMAT
can be seen by examining the current code base.  If you want a structured view into these details,
you should run the open source tool Doxygen\cite{doxygen} on the source code tree. Doxygen will
create an extensive hyperlinked reference for the GMAT code that you can browse using any HTML
browser.


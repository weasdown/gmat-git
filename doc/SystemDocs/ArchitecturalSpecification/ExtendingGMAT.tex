% $Id: ExtendingGMAT.tex,v 1.1 2008/01/31 18:04:16 dconway Exp $
\chapter{\label{chapter:ExtendingGMAT}Adding New Objects to GMAT}
\chapauthor{Darrel J. Conway}{Thinking Systems, Inc.}

Chapter~\ref{chapter:Factories} provided an introduction to the GMAT Factory subsystem.  This
feature of the GMAT design provides an interface that users can use to extend GMAT without
impacting the core, configuration managed, code base.  Any of the scriptable object types in the
system can be extended using this feature; this set of objects includes hardware elements,
spacecraft, commands, calculated parameters, and any other named GMAT objects.  This chapter
provides an introduction to that interface into the system.
\section{Shared Libraries}

\section{Adding Classes to GMAT}

\subsection{Designing Your Class}

This is a list of steps taken to construct the steepest descent solver.

\begin{itemize}
\item Create the class (.cpp and header, comment prologs, etc.).
\item Add shells for the abstract methods.
\item Fill in code for the shells.
\item Add the object file to the list of objects in the (base) makefile.
\item Unit test if possible.
\item Build the code and debug what can be accessed at this point.
\end{itemize}

\subsection{Creating the Factory}

This is a list of steps taken to incorporate the steepest descent solver.

\begin{itemize}
\item Create the factory (in this case I edited SolverFactory).
\item Add constructor call to the appropriate ``Create...'' method.
\item Add the new object type name to the ``creatables'' lists in the factory constructors.
\item Build and fix any compile issues.
\item Test to see if the object can be created from a script.
\end{itemize}

\subsection{Bundling the Code}

\subsection{Registering with GMAT}

\section{An Extensive Example}

%-----------------------------------------------------------------------
%-----------------------Begin Table Here--------------------------------
%----- (Comment the \clearpage code if the table doesn't break ----------
%------ across multiple pages or is the first in a section) ------------
%-----------------------------------------------------------------------
%\clearpage
\noindent
\tablecaption{Assignment Command}
\tablefirsthead{\hline\hline}\label{Table:AssignmentCommand}\index{\st{Assignment}}
\tablehead{\multicolumn{2}{c}{Table~\ref{Table:AssignmentCommand}: Assignment Command \ldots
continued}\\\\ \hline\hline}
\tabletail{\hline\hline}
\tablelasttail{\hline\hline}
\begin{supertabular*}{6.5 in}{@{}p{1.5 in}@{\extracolsep{\fill}}p{5.0 in}@{}}
    \multicolumn{2}{l}{}\\
    \multicolumn{2}{l}{Script Syntax: \st{GMAT Arg1 = Arg2;}}\\\\
    \hline\hline
    Command Description\\
    \hline
    %------- New Item
    \st{Arg1} & Default: N/A . Options:[Spacecraft Parameter, Array element, Variable,
    or any other single element user defined parameter]:
    %  Description
    The \st{Arg1} option allows the user to set \st{Arg1} to \st{Arg2}. Units: N/A.\\
    \st{Arg2} & Default: N/A . Options:[Spacecraft Parameter, Array element, Variable,
    any other single element user defined parameter, or a combination of the aforementioned
    parameters using math operators]:
    %  Description
    The \st{Arg2} option allows the user to define \st{Arg1}. Units: N/A.\\\\

    \hline\hline
    \multicolumn{2}{c}{Script Examples}\\
    \hline
    \multicolumn{2}{l}{\% Setting a variable to a number}\\
    \multicolumn{2}{l}{\st{GMAT testVar = 24;}}\\
    \multicolumn{2}{l}{\% Setting a variable to the value of a math statement}\\
    \multicolumn{2}{l}{\st{GMAT testVar = (testVar2 + 50)/2;}}\\\\

\end{supertabular*}\\

\subsection{$n$-Body Point Mass Gravity}
In this section, we derive the gravitational acceleration arising from $n$ bodies modelled as point masses. We derive the governing differential equations, as
well as the Jacobians for position, velocity, mass, and time. 

Let's begin by defining some notation referring to
Fig.\ref{fig:NBody}. Assume the $j^{\mbox{th}}$ body is the central
body of integration.
%
\begin{figure}[h!]
\centerline{
\begin{picture}(100,500)
\special{psfile= Images/NBodyDiagram.eps hoffset= -135 voffset= -45
hscale=85 vscale=85} \makebox(-20,585){$\hat{\mathbf{x}}_{I}$}
\makebox(270,700){$\hat{\mathbf{y}}_{I}$} \makebox(-330,770){$
\tilde{\mathbf{r}}_{s}$} \makebox(-330,877){$\mathbf{r}$}
\makebox(-270,900){$\mathbf{r}_{sk}$}
\makebox(-390,964){$\mathbf{r}_{k}$}
\makebox(-500,814){$\tilde{\mathbf{r}}_{j}$}
\makebox(-500,980){Central Body} \makebox(-320,995){$k^{th}$ Body}
\end{picture}}\vskip -4.0 in  \caption{ N-Body Illustration} \label{fig:NBody}
\end{figure}
%
%
\begin{itemize}
   %
   \item  $\tilde{\mathbf{r}}_s$ is the position of the spacecraft with respect
   a hypothesized inertial frame.
   %
   \item  $\tilde{\mathbf{r}}_j$ is the position of the central body with respect
   a hypothesized inertial frame.
   %
   \item  $\tilde{\mathbf{r}}_k$ is the position of the $k^{th}$ gravitational body with respect
   a hypothesized inertial frame.
   %
   \item  $\mathbf{r}$ is the position of the spacecraft with respect
   to the central body of integration ($j^{th}$ body).
   %
   \item  $\mathbf{r}_k$ is the position of the $k^{th}$ gravitational body with respect
   to the central body.
   %
\end{itemize}

We begin by defining the relative position of the spacecraft
with respect to the central body.  From inspection of
Fig.\ref{fig:NBody} we see that
%
\begin{equation}
     \tilde{\mathbf{r}}_j +  \mathbf{r} = \tilde{\mathbf{r}}_s
\end{equation}
%
By reordering and taking the second derivative with respect to time
we obtain
%
\begin{equation}
     \ddot{\mathbf{r}} = \ddot{\tilde{\mathbf{r}}}_s - \ddot{\tilde{\mathbf{r}}}_j
     \label{Eq:SCRelativeODE}
\end{equation}
%
We can apply Newton's 2nd Law to the spacecraft and obtain
%
\begin{equation}
     m_s \ddot{\tilde{\mathbf{r}}}_s = \sum_{k=1}^n F_k =
     G\sum_{k=1}^n \frac{m_s m_k}{\| \mathbf{r}_{k} - \mathbf{r}\|^3} \left(\mathbf{r}_{k} -
     \mathbf{r}\right)
\end{equation}
%
where $\left(\mathbf{r}_{k} - \mathbf{r}\right)$ is a vector from
the spacecraft to the $k^{th}$ body, $m_s$ is the mass of the
spacecraft, and $m_k$ is the mass of the $k^{th}$ body.  We can
write $\ddot{\tilde{\mathbf{r}}}_s$ as simply
%
\begin{equation}
    \ddot{\tilde{\mathbf{r}}}_s =
     G\sum_{k=1}^n \frac{m_k}{\| \mathbf{r}_{k} - \mathbf{r}\|^3} \left(\mathbf{r}_{k} -
     \mathbf{r}\right) \label{Eq:SCInertialODE}
\end{equation}
%
We can apply Newton's 2nd Law to the $j^{th}$ body and obtain
%
\begin{equation}
     m_j \ddot{\tilde{\mathbf{r}}}_j = \frac{G m_s
     m_j}{r^3}\mathbf{r} +
     G\sum_{\stackrel{k=1}{k \neq j}}^{n} \frac{m_j m_k}{\| \mathbf{r}_{k}\|^3}\mathbf{r}_{k}
\end{equation}
%
where the first term is the influence of the spacecraft on the
central body, and the second term is the influence of the $k$ point
mass gravitational bodies.  We can write
$\ddot{\tilde{\mathbf{r}}}_j$ as simply
%
\begin{equation}
     \ddot{\tilde{\mathbf{r}}}_j = \frac{G m_s
     }{r^3}\mathbf{r} +
     G\sum_{\stackrel{k=1}{k \neq j}}^{n} \frac{ m_k}{\|
     \mathbf{r}_{k}\|^3}\mathbf{r}_{k} \label{Eq:CentalBodyInertialODE}
\end{equation}
%
Substituting Eq.~(\ref{Eq:SCInertialODE}) and
(\ref{Eq:CentalBodyInertialODE}) into (\ref{Eq:SCRelativeODE}) we
get
%
\begin{equation}
     \ddot{\mathbf{r}} =      G\sum_{k=1}^n \frac{m_k}{\| \mathbf{r}_{k} - \mathbf{r}\|^3} \left(\mathbf{r}_{k} -
     \mathbf{r}\right) - \frac{G m_s
     }{r^3}\mathbf{r} -
     G\sum_{\stackrel{k=1}{k \neq j}}^{n} \frac{ m_k}{\|
     \mathbf{r}_{k}\|^3}\mathbf{r}_{k}
\end{equation}
%
Finally, collecting terms yields
%
\begin{equation}
     \mathbf{\ddot{r}}_{pm} = \underbrace{- \frac{\mu_j
     }{r^3}\mathbf{r}}_1  +  G \sum_{\stackrel{k=1}{k \neq j}}^{n} m_k\left( \underbrace{\frac{\mathbf{r}_{k} -
     \mathbf{r}}{\| \mathbf{r}_{k} - \mathbf{r}\|^3}}_2  -
     \underbrace{
      \frac{ \mathbf{r}_{k}}{\|
     \mathbf{r}_{k}\|^3}}_3\right)
\end{equation}
%
We can break down the acceleration in the equation above into three
physical categories.   The first term is the acceleration on the
spacecraft due to a point mass central body.   The second type of
terms are called direct terms.  They account for the force of the
$k^{th}$ body on the spacecraft.  The third type of terms are called
indirect.  They account for the force of the $k^{th}$ body on the
central body.

The force is conservative so velocity and mass partials are zero.
%
\begin{equation}
    \frac{\partial \mathbf{\ddot{r}}_{pm}}{\partial \mathbf{v}}  = \mathbf{0}_{3\times3}
\end{equation}
%
%
\begin{equation}
    \frac{\partial \mathbf{\ddot{r}}_{pm}}{\partial m}  = \mathbf{0}_{3\times1}
\end{equation}
%
To determine the partials with respect to position, we can use the vector identity in
Eq.~(\ref{Eq:vecIDaveca3}) to arrive at
%
\begin{equation}
     \frac{\partial }{\partial \mathbf{r}} \left(- \frac{\mu_j
     }{r^3}\mathbf{r}\right) =  -\frac{\mu_j}{r^3} \mathbf{I}_3
     + 3\mu_j\frac{\mathbf{r}\mathbf{r}^T}{r^5}
\end{equation}

Similarly, applying Eq.~(\ref{Eq:vecIDaveca3}) to the direct terms
we see that
%
\begin{equation}
     \frac{\partial }{\partial \mathbf{r}} \left( \sum_{\stackrel{k=1}{k \neq j}}^{n} \mu_k \frac{\mathbf{r}_{k} -
     \mathbf{r}}{\| \mathbf{r}_{k} - \mathbf{r}\|^3}\right) =  -\sum_{\stackrel{k=1}{k \neq j}}^{n}
     \frac{\mu_k}{\| \mathbf{r}_{k} - \mathbf{r}\|^3}\mathbf{I}_3 + 3\sum_{\stackrel{k=1}{k \neq
     j}}^{n}\mu_k \left( \frac{\left( \mathbf{r}_{k} - \mathbf{r} \right)\left( \mathbf{r}_{k} -
     \mathbf{r} \right)^T}{\left( \|\mathbf{r}_{k} - \mathbf{r} \right)\|^5}  \right)
\end{equation}
%
Finally, the derivative of the indirect terms are zero and we have
%
\begin{equation}
   \frac{\partial \mathbf{\ddot{r}}_{pm}}{\partial \mathbf{r}} =  \underbrace{-\frac{\mu_j}{r^3} \mathbf{I}_3
     + 3\mu_j\frac{\mathbf{r}\mathbf{r}^T}{r^5}}_{ 1 }
     \underbrace{
     %
     -  \sum_{\stackrel{k=1}{k \neq j}}^{n}
     \frac{\mu_k}{\| \mathbf{r}_{k} - \mathbf{r}\|^3}\mathbf{I}_3 + 3\sum_{\stackrel{k=1}{k \neq
     j}}^{n} \mu_k \left( \frac{\left( \mathbf{r}_{k} - \mathbf{r} \right)\left( \mathbf{r}_{k} -
     \mathbf{r} \right)^T}{\left( \|\mathbf{r}_{k} - \mathbf{r} \right)\|^5}  \right)
       }_{2}
     %
\end{equation}
%
Combining similar terms we can express the result as
%
\begin{equation}
   \frac{\partial \mathbf{\ddot{r}}_{pm}}{\partial \mathbf{r}} =   -  \left( \frac{\mu_j}{r^3} + \sum_{\stackrel{k=1}{k \neq j}}^{n}
     \frac{\mu_k}{\| \mathbf{r}_{k} - \mathbf{r}\|^3} \right)\mathbf{I}_3
     %
     + 3 \left( \mu_j\frac{\mathbf{r}\mathbf{r}^T}{r^5}
       + \sum_{\stackrel{k=1}{k \neq
     j}}^{n} \mu_k \left( \frac{\left( \mathbf{r}_{k} - \mathbf{r} \right)\left( \mathbf{r}_{k} -
     \mathbf{r} \right)^T}{\left( \|\mathbf{r}_{k} - \mathbf{r} \right)\|^5}
     \right) \right)
\end{equation}
%
The time Jacobian is 
%
\begin{equation}
    \frac{\partial \mathbf{\ddot{r}}_{pm}}{\partial t} = \sum_{k=1}^{n}\mu_k\left(
%    
    \frac{1}{r_{rel}^{3}}\left( \mathbf{I } - 3 \hat{\mathbf{r}}_{rel}\hat{\mathbf{r}}_{rel}^{T}\right)
%    
-
\frac{1}{r_{k}^{3}}\left( \mathbf{I } - 3 \hat{\mathbf{r}}_{k}\hat{\mathbf{r}}_{k}^{T}\right)
     \right)\mathbf{v}_k
\end{equation}
\onecolumn
\chapter{Measurement Models}


\begin{equation}
    \mathcal{R}_1^{(j)} = \rho^{(m,j)}_1(t_i) +c\left(b_r^{(m)}(t_i) - b_t^{(j)}(t_t)\right) + \Delta \rho^{(j)}_{iono}(t_i) + \Delta \rho^{(j)}_{tropo}(t_i) + \sigma_{SA}^{(j)} + \sigma_n^{(j)} + b_m
\end{equation}

\begin{tabbing}
12345678912345 \= Reynolds number based on length $s$ \kill
$\mathcal{R}_1^{(j)}$         \>  One-way pseudorange measurement from transmitting \\
$$                            \> antenna $k$ on satellite $j$ to receiving antenna $m$ on satellite $n$. \\
$\rho^{(m,j)}_1(t_i)$         \>  Geometric distance between transmitting and receiveing antenntas.\\
$b_r^{(m)}(t_i)$               \> Clock bias for receiver \\
$b_t^{(j)}(t_t)$                   \>  Clock bias for transmitter\\
$\Delta \rho^{(j)}_{iono}(t_i)$    \> Correction for ionspheric delay \\
$ \Delta \rho^{(j)}_{tropo}(t_i) $ \> Correction for  Tropospheric delay\\
$\sigma_{SA}^{(j)}$                \> Error due to selective availability \\
$\sigma_n^{(j)}$                   \> Measurement noise \\
$b_m$                              \> Measurement bias \\
$t_t$                              \> Trasmission time \\
$t_r$                              \> Receive time
\end{tabbing}


The transmission time $t_t$ and the geometric range are determined
iteratively by using fixed point iteration on the following equation
%
\begin{equation}
   t_t = t_i - \frac{\rho^{(m,j)}_1(t_i) + \Delta \rho^{(j)}_{iono}(t_i) + \Delta \rho^{(j)}_{tropo}(t_i)}{c}
\end{equation}

\section{Range}

GMAT supports two primary models for spacecraft range: geometric and
radiometric. In both cases the range is a measure of the distance
between an observer and a vehicle.
 The geometric range is calculated using vector geometry and ignores signal propagation
  times and error sources.  Most, if not all, ground trackers provide the user with
  the round trip signal propagation time.  The radiometric range model uses the best
   estimate spacecraft state to determine an expected value for the round trip signal
   propagation time from observer to spacecraft and back to the observer.  Hence,
   the raw radiometric value is a measure of round trip range.

The general form of the measurement model used by GMAT is
%
\begin{equation}
   \mathcal{O}_c = \mathbf{f}_k\left(\mathbf{r}_v(t + \delta t), \dot{\mathbf{r}}_v(t + \delta t),
   \mathbf{r}_o, \dot{\mathbf{r}}_o\right) + b + \delta_a + \delta_r
\end{equation}
%
%
\begin{tabbing}[htbp!]
12345678 \= dummy line \kill
$\mathbf{f}_k$ \> The kinematic model specific to a measurement type\\
$\mathbf{r}_v$ \> Vehicle position\\
$\dot{\mathbf{r}}_v$ \> Vehicle velocity\\
$\mathbf{r}_o$ \> Observer position\\
$\dot{\mathbf{r}}_o$    \> Observer velocity\\
$t$    \> Measurement time tag\\
$b$    \> Measurement bias\\
$\delta a$    \> Atmospheric correction\\
$\delta r$     \> Relativistic correction\\
\end{tabbing}

The kinematic model for geometric range is simply
%
\begin{equation}
    \rho_c = \| \mathbf{r}_v(t) - \mathbf{r}_g(t)\|
\end{equation}
%

The kinematic model for radiometric two way range is
%
\begin{equation}
     \rho_c= \frac{1}{2}\left(\| \mathbf{r}_v(t_{v}) -  \mathbf{r}_g(t_{gt})  \| +
      \| \mathbf{r}_v(t_{v}) -  \mathbf{r}_g(t_{gt})  \|\right) \label{Eq:ExpectedTwoWayRange}
\end{equation}
%
where
%
\begin{tabbing}[htbp!]
12345678 \= dummy line \kill
$t_{gt}$ \> Time the uplink signal is transmitted from ground station\\
$t_{vr}$ \> Time the uplink signal is received at vehicle\\
$t_{vt}$ \> Time the downlink signal is transmitted from vehicle\\
$t_{gr}$ \> Time the downlink signal is received at ground station\\
$t_u$    \> Uplink propagation time, ( $t_{vr}$ - $t_{gt}$ )\\
$t_d$    \> Downlink propagation time, ( $t_{vt}$ - $t_{gr}$ )\\
$\rho_u$    \> Uplink distance\\
$\rho_d$    \> Downlink distance\\
$d_T$     \> Distance traveled by vehicle during transponder delay\\
$\mathbf{r}_v(t)$ \> Position of vehicle at time $t$\\
$\mathbf{r}_t(t)$ \> Position of transmitter at time $t$\\
$\mathbf{r}_r(t)$ \> Position of receiver at time $t$\\
$t$           \>  time of geometric range measurement \\
$\delta T$  \>  Vehicle's transponder delay time\\
$\delta t_a$  \>  Atmospheric delays\\
$\delta t_r$  \>  Relativistic effects\\
\end{tabbing}
%

The radiometric model is derived from the measurement geometry shown
in Fig.~\ref{Fig:RangeMeasurement}. We see that the total signal
propagation time is the sum of three times, the uplink
 signal propagation time, $t_u$, the transponder delay time, $\delta T$, and
 the downlink propagation time, $t_d$.  Hence, the observed value for signal propagation time, $t_o$, is
%
\begin{equation}
     \Delta t_o = t_{gr} - t_{gt} = t_u + \delta T + t_d
\end{equation}
%
\begin{figure}[htbp!]
    \begin{center}
    \begin{picture}(270,215)
    \special{psfile= RangeMeasurement.eps
    hscale= 75 vscale= 75 hoffset = -85 voffset = -270}
        \makebox(175,295){ $\mathbf{r}_v(t_{vr})$}
        \makebox(-25,290){ $\mathbf{r}_v(t_{vt})$}
        \makebox(85,290){ $\rho_d$}
        \makebox(-100,90){ $\mathbf{r}_g(t_{gr})$}
        \makebox(-460,90){ $\mathbf{r}_g(t_{gt})$}
        \makebox(-485,290){ $\rho_u$}
        \makebox(-305,390){ $d_T$}
    \end{picture}
    \end{center}
    \vspace{0.2 in}
    \label{Fig:RangeMeasurement}
    \caption{ Geometry of Radiometric Range Measurement}
\end{figure}
%
The elapsed time is converted to a measure of the average range
using
%
\begin{equation}
     \rho_o = \frac{c}{2}\Delta t_o
\end{equation}
%
The computed elapsed time is rigorously expressed as
%
\begin{equation}
    \Delta t_c = \frac{1}{c}\| \mathbf{r}_v(t_{vr}) -  \mathbf{r}_g(t_{gt})  \| +
    \frac{1}{c}\| \mathbf{r}_v(t_{vt}) -  \mathbf{r}_g(t_{gr})  \| + \delta T
\end{equation}
%
Assuming that the transponder delay is modelled as a measurement
bias we can write $ t_{vr} = t_{vt} = t_v$ and convert the computed
round trip time to average range:
%
\begin{equation}
     \rho_c= \frac{1}{2}\left(\| \mathbf{r}_v(t_v) -  \mathbf{r}_g(t_{gt})  \| +
      \| \mathbf{r}_v(t_v) -  \mathbf{r}_g(t_{gr})  \|\right) \label{Eq:MeasuredTwoWayRange}
\end{equation}
%
To solve Eq.~\ref{Eq:MeasuredTwoWayRange}, we must know  $t_{v}$.
For applications
 that do not require high accuracy we can approximate $t_v$ as we describe below.
  For higher fidelity applications we must solve for the uplink and downlink
  propagation times using two iterative processes.  The downlink signal propagation
  time is calculated using the following fixed point iteration on the following equation:
%
\begin{equation}
     \delta t_d^{i+1} = \frac{1}{c}\| \mathbf{r}_v( t - t_d^{i}) - \mathbf{r}_g(t)   \|
\end{equation}
%
The uplink propagation time is calculated using fixed point
iteration on
%
\begin{equation}
     \delta t_u^{i+1} = \frac{1}{c}\| \mathbf{r}_v( t - t_d) - \mathbf{r}_g(t - t_d - t_u^{i} )   \|
\end{equation}
%

\begin{eqnarray}
     \frac{\partial \rho_c (t)}{\partial \mathbf{r}_v(t_v)} &=& \frac{1}{2\rho_u\rho_d}
     \left( \rho_d(\mathbf{r}_v^T(t_v) - \mathbf{r}_T^T(t_{gt}) + \rho_u(\mathbf{r}_v^T(t_v) - \mathbf{r}_R^T(t_{gr}  ) \right)\\
     %
     \frac{\partial \rho_c (t)}{\partial \dot{\mathbf{r}}_v(t_v)} &=& \mathbf{0}_{1x3}
\end{eqnarray}

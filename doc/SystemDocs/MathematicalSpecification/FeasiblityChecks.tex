\section{Measurement Editing and Feasibility Criteria}

\subsection{Line of Site Test}

This test checks to see if a celestial body obstructs the signal path between two objects.  There are three cases that can occur in this test: two spacecraft (case 1), a spacecraft and a participant on the surface of a celestial body (case 2), and two participants on the surfaces of different celestial bodies (case 3).  Below we address each of these cases starting with case 1.

For all three cases, assume the signal is generated by participant 1 at time $t_1$ and, if no obstruction occurs, the signal is received by participant 2 at time $t_2$. Define the location of the first participant at time $t_1$, expressed in $\mcF_{\mcI_1}$ as $\mathbf{r}_1^{\mcI_1}(t_1)$ which is calculated using Eq.~(\ref{Eq:LocalInertialAntennaPos1}). Define the location of participant 1 at time $t_1$, expressed in $\mcF_{\mcS}$, as $\mathbf{r}_1^{\mcS}(t_1)$ which is calculated using Eq.~(\ref{Eq:BaryInertialAntennaPos1}). Similarly, $\mathbf{r}_2^{\mcI_2}(t_2)$ and $\mathbf{r}_2^{\mcS}(t_2)$ are calculated using Eqs.~(\ref{Eq:LocalInertialAntennaPos2}) and (\ref{Eq:BaryInertialAntennaPos2}) respectively.


\subsubsection{Two Spacecraft}





\subsection{Height of Ray Path}



\subsection{Line of Sight}

The Line of Sight (LOS) test is a visibility test for inter-spacecraft measurements.
The algorithm presented here is based on Vallado\cite{vallado3} (pp. 307-311) with slight modifications to include light time correction when applicable (i.e. for all measurements except geometric measurements). This test first checks to see if the intersection of the perpendicular distance vector, $\mathbf{d}(\tau_{min})$,  with the ray path, $\boldsymbol{\rho}$, lies in between the two participants. It then checks to see if the ray path height, $h$, is above the minimum allowable ray path altitude $h_{min}$.


The geometry for the LOS test is shown in Fig.~({\ref{Fig:HORP}).
%
\begin{figure}[h!]
    \begin{center}
        \begin{picture}(270,145)
            \special{psfile= ./Images/HORP.eps hscale= 100 vscale= 100 hoffset = -45 voffset = -455}
             \makebox(115,15){${h}_{min}$}
             \makebox(35,20){$R_b$}
             \makebox(60,90){$\mathbf{r}_2$}
             \makebox(-150,150){$d$}
             \makebox(-150,210){$\boldsymbol{\rho}$}
             \makebox(-310,150){$\mathbf{r}_1$}
        \end{picture}
    \end{center}
    \vspace{.2 in}
    \caption{ Height of Ray Path Geometry }
    \label{Fig:HORP}
\end{figure}
%
The variables in Fig.~\ref{Fig:HORP} are defined as follows.
%
\begin{center}
    \begin{minipage}[t]{5.0 in}
        \begin{tabbing}[htbp!]
            123456 \= dummy line \kill
            $\mathbf{r}_1(t_1)$ \> Position vector of participant 1 at time of signal transmission, $t_1$ \\
            $\mathbf{r}_2(t_2)$ \> Position vector of participant 2 at time of signal reception, $t_2$ \\
            $\boldsymbol{\rho}$ \> Vector from location of particpant 1 at signal transmission to location\\
             \>  of participant 2 at signal reception\\
            $\rho$ \> Magnitude of $\boldsymbol{\rho}$ \\
            $R_b$ \> Central body radius \\
            $h_{min}$ \> Minimum acceptable ray path height \\
            $\tau$ \> Ray path parameter defined as the fraction of the unit distance along $\boldsymbol{\rho}$ \\
            $\mathbf{d}(\tau)$ \> Position vector of $\tau$ \\
            $\tau_{min}$ \> Value of $\tau$ minimizing the distance between $\boldsymbol{\rho}$ and the central body \\
            $d(\tau_{min})$ \> Distance from central body origin to ray path (measured perpendicular) \\
        \end{tabbing}
    \end{minipage}
\end{center}
%
The ray path vector, $\boldsymbol\rho$, is computed from
%
\begin{equation}
   \left[\boldsymbol{\rho}\right]_1 =  \left[ \mathbf{r}_2(t_2) \right]_1 -  \left[ \mathbf{r}_1 (t_1) \right]_1
\end{equation}
%
where the quantities $\mathbf{r}_2(t_2)$ and $\left[ \mathbf{r}_1 (t_1) \right]_1$ are determined during light time correction.
The ray path parameter, $\tau$, is a unitless value indicating a point along $\boldsymbol{\rho}$. The only value of $\tau$ that we care about here is $\tau_{min}$, which is computed from
%
\begin{equation}
     \tau_{min} = \frac{\left[ \mathbf{r}_2 (t_2) \right]_1 \cdot \left[\boldsymbol{\rho}\right]_1} {\rho^2}
\end{equation}
%
The position vector of $\tau$, $\mathbf{d}(\tau)$, is simply
%
\begin{equation}
    \mathbf{d}(\tau) = \left[ \mathbf{r}_2(t_2) \right]_1 -  \left[\boldsymbol{\rho}\right]_1 \tau
\end{equation}
%


Finally, the criteria for measurement feasibility is when the following statement is true
%
\begin{equation}
    T^2 - T > 0 \hspace{.05 in} \mbox{or} \hspace{.05 in} d(\tau_{min})^2 - R_b^2 >= 0
\end{equation}
%
For geometric measurements, $t_1 = t_2$.  For measurements involving participants about different central bodies, the LOS test is performed twice, once for each central body.

\subsection{Horizon Angle Test}

The horizon angle test checks to see if a space-based observer  is above the local horizon for a ground-based observer. The space-based observer can be a spacecraft or ground based observer on another celestial body.  The geometry for this test is shown in Fig. \ref{Fig:HorizonAngle} where the nomenclature is defined below.
%
\begin{figure}[h!]
    \begin{center}
        \begin{picture}(270,125)
            \special{psfile= ./Images/HorizonTest.eps hscale= 100 vscale= 100 hoffset = -45 voffset = -495}
             \makebox(290,100){$\mathbf{r}_s$}
             \makebox(-360,100){$\mathbf{r}_g$}
             \makebox(-300,200){$\boldsymbol{\rho}$}
             \makebox(-215,105){$\delta_m$}
             \makebox(-190,195){$\delta$}
        \end{picture}
    \end{center}
    \vspace{.02 in}
    \caption{ Height Angle Test Geometry }
    \label{Fig:HorizonAngle}
\end{figure}
%
\begin{center}
    \begin{minipage}[t]{5.0 in}
    \begin{tabbing}[htbp!]
            123456 \= dummy line \kill
            $\mathbf{r}_g$ \> Position vector of the ground-based observer at time $t_g$\\
            $t_g$ \> Time the signal is at the receiving electronics of the ground-based observer \\
            $\mathbf{r}_s$ \> Position vector of the space-based observer at time $t_s$ \\
            $t_s$ \> Time the signal is at the receiving electronics of the space-based observer \\
            $\boldsymbol{\rho}$ \> Vector from location of ground-based observer to  space-based observer\\
             \>  of participant 2 at signal reception\\
            $\delta_m$ \> Minimum elevation angle above local horizon of ground-based observer\\
        \end{tabbing}
    \end{minipage}
\end{center}
%
From inspection of Fig. \ref{Fig:HorizonAngle}, a space-based observer is above the local horizon
of a ground-based observer if
%
\begin{equation}
    \delta \geq \delta_m
\end{equation}
%
From manipulation of the inner product we know
%
\begin{equation}
    \sin{\delta} = \frac{ \boldsymbol{\rho} \cdot \mathbf{r}_g }{\rho r_g }
\end{equation}
%
where $\boldsymbol{\rho} = \mathbf{r}_s - \mathbf{r}_g$. Finally, the horizon check is true if the following statement is satisfied and false otherwise:
%
\begin{equation}
     \frac{ \boldsymbol{\rho} \cdot \mathbf{r}_g }{\rho r_g } > \sin{\delta_m}
\end{equation}


\subsection{Range Limit Test}
\subsection{Range Rate Limit Test}
\subsection{Solar Exclusion Angle Test}

\section{Project and System Overview}

\subsection{Objectives and Goals}

The goal of the GMAT project is to develop new astrodynamics technologies and
provide
software for operational mission design and navigation support
by working inclusively with individuals, universities, businesses, and other
government organizations.  A second and important goal is to share that
technology in an open and unhindered way. GMAT has been approved for release
under the NASA Open Source Agreement (NOSA). You, your business, or your
organization can get involved in the GMAT project in numerous ways. We use an
open source model to encourage collaboration and to maximize technology
transfer.  Individuals, universities, industry, and other government
organizations can contribute and collaborate in ways that meet their respective
goals, needs, and interests.

\subsection{Contributors}

GMAT contributors include volunteers and those paid for services they provide.
We welcome new contributors to the project, either as users providing feedback
about the features of the system, or as developers interested in contributing to
the implementation of the system.  Current U.S. government participants include
NASA and the Air Force Research Lab (AFRL).  Past and present industry contributors to GMAT
include Thinking Systems, Inc. (system architecture and all aspects of
development), a.i.-solutions (testing), Boeing (algorithms and testing), The
Schafer Corporation (all aspects of development), Honeywell Technology Solutions
(testing), and the Computer Sciences Corporation (requirements). The NASA Jet
Propulsion Laboratory (JPL) is providing funding for integration of their SPICE
toolkit into GMAT. Additionally , the European Space Agency's (ESA) advanced concepts team has
developed optimizer plug-ins for the Non-Linear Programming (NLP) solvers SNOPT (Sparse Nonlinear OPTimizer) and IPOPT (Interior Point OPTimizer) using the process described in
the later half of this paper.

%Participants from three branches at GSFC are involved in GMAT development.
The Navigation and Mission Design Branch at NASA's Goddard Space Flight Center performs project management activities and is
involved in most phases of the development process including requirements,
algorithms, design, and testing. The Ground Software Systems Branch performs
design, implementation, and integration testing. The Flight Software Branch
contributes to design and implementation.  AFRL is involved in all development
areas but primarily in the areas of requirements specification, mathematical and
algorithmic specifications, system prototyping, and testing.

\subsection{Platforms}

GMAT is written to run on Windows, Linux and Macintosh platforms, using the
wxWidgets cross platform UI Framework, and can be built using either commercial
development tools or the GNU Compiler Collection (GCC). The system is
implemented in ANSI standard C++ (approximately 250,000 non-comment source lines
of code) using an Object Oriented methodology, with a rich class structure
designed to make new features simple to incorporate.

On Windows and Linux, GMAT does not call any operating system unique functions
or methods. Calls to the operating system are standard calls for reading and
writing data files and for writing data to the screen.  On the Mac, GMAT makes
a call to the operating system to open X11, which is required to run MATLAB on
the Mac.

\subsection{User Interfaces}

GMAT has three user interfaces. The graphical user interface (GUI) illustrated
in Figs.~(\ref{fig:ResourceTreeCapture}), (\ref{fig:MissionTreeCapture}), and
(\ref{fig:LunarScreenCapture}) allows the user to set up and execute all aspects
of GMAT.  The script interface is textual and also allows the user to set up and
execute all aspects of GMAT. The MATLAB interface is a secondary textual interface
for running the system via calls from GMAT to MATLAB.

\subsection{Status}

While GMAT has undergone extensive testing and is mature software, at the present time we consider
the software to be in Beta form.  GMAT is not yet sufficiently verified to be
used as a primary operational analysis sytem. GMAT has been used to optimize maneuvers for flight projects such as NASA's LCROSS and ARTEMIS missions, and for formation optimization and analysis for the
MMS mission.  However, for flight planning, we independently verify solutions
generated in GMAT in the primary operational system.

The GMAT Team is currently working on several activities including maintenance,
bug fixes, and the implementation of estimation components.   The objective of
the current development cycle is to provide a stable, non-beta release in the fall of
2010.







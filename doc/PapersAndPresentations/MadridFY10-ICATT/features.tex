
\section{Features}

%interfaces overview.. figure?
%
%script- text based
%gui
%resources and commands
%
%List of resources
%List of commands
%
%user workflow
%main areas of application
%optimization
%maneuver design planning
%monte carlo
%estimation

GMAT is designed to model, optimize, and estimate spacecraft trajectories in flight regimes ranging from low Earth orbit to lunar applications, interplanetary trajectories, and other deep space missions.

Analysts model space missions in GMAT by first creating resources such as spacecraft, propagators, estimators, and optimizers.  A figure of the resource tree for a lunar transfer application is shown in Fig. (\ref{fig:ResourceTreeCapture}).
%
\begin{figure}[tb]
\begin{center}
\includegraphics*[scale=0.58]{Images/ResourceTreeScreenShot.eps}
\caption{\label{fig:ResourceTreeCapture} Screen Capture of Resource Tree}
\end{center}
\end{figure}
%
Resources can be configured to meet the needs of specific applications and missions.   GMAT contains an extensive set of available Resources that can be broken down into physical model Resources and analysis model Resources.  Physical Resources include  spacecraft,  thruster, tank, transmitter*, transponder*, antenna*, receiver*, ground station, formation, impulsive burn, finite burn, planet, comet, asteroid, moon, barycenter, libration point, measurement model*,  and measurement simulator*. Analysis model Resources include differential corrector, propagator, optimizer ,estimator*, 3-D graphic, x-y plot, report file, ephemeris file, user-defined variable, array, and string, coordinate system, custom subroutine, MATLAB function, and data file*.  (Items with ``*" are currently under development and not available in the public repository at the time of this writing as they have not been reviewed for ITAR and other release issues.)

After the resources are configured, they are used in the mission sequence, as shown in Fig.~(\ref{fig:MissionTreeCapture}) to model spacecraft motion and simulate events in a mission's time evolution.  Users employ built-in Commands that simulate trajectory dynamics or apply numerical methods such as estimators, optimizers, and boundary value solvers. The mission sequence supports the following commands: propagate, impulsive maneuver, finite maneuver, target, optimize, estimate, simulate measurements, non-linear constraint, minimize, call functions, inline math, vary parameter, achieve parameter,  if/else, for, and while, and report.

The system can display trajectories in space, plot parameters against one another, and save parameters to files for later processing.  The trajectory and plot capabilities are fully interactive, plotting data as a mission is run and allowing users to zoom into regions of interest.
%
\begin{figure}[tb]
\begin{center}
\includegraphics*[scale=0.58]{Images/MissionTreeScreenShot.eps}
\caption{\label{fig:MissionTreeCapture} Screen Capture of the Mission Tree}
\end{center}
\end{figure}

Trajectories and data can be viewed in any coordinate system defined in GMAT, and GMAT allows users to rotate the view and set the focus to any object in the display.  The trajectory view can be animated so users can watch the evolution of the trajectory over time.  A screen capture of the graphics after computing a lunar transfer is shown in Fig.~(\ref{fig:LunarScreenCapture}).
%
\begin{figure}[tb]
\begin{center}
\includegraphics*[scale=0.225]{Images/LunarTransferScreenShot.eps}
\caption{\label{fig:LunarScreenCapture} Screen Capture of Lunar Transfer in GMAT}
\end{center}
\end{figure}

% The system supports constrained and unconstrained trajectory optimization and built-in features make defining cost %and constraint functions trivial, so analysts can determine how their inclusion or exclusion effects solutions.
%The system also contains initial value solvers (propagation) and boundary value solvers and efficiently propagates spacecraft either singly or coupled.   GMAT's propagators naturally synchronize the epochs of multiple vehicles and shorten run times by avoiding fixed step integration or interpolation to synchronize epochs of spacecraft.
%A user can interact with GMAT using either a graphical user interface (GUI) or  script language that has a syntax similar to the MathWorks' MATLAB\textregistered system.  All of the system elements can be expressed through either interface and users can configure elements in the GUI and then view the corresponding script, or write script and load it into GMAT. 